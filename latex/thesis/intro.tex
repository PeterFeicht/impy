\chapter{Introduction}

The goal is to develop a compact and affordable way of measuring impedance spectra. Commercially available impedance
analyzers, like the Agilent 4294A, are bulky and rather expensive devices, which are ofter overkill for simple
tasks, such as monitoring biological samples over many days or weeks. When the requirements for accuracy or the covered
frequency range are not very strict, such an instrument can be replaced by the simple device presented here.

The finished device is small, it fits on a single \unit{100 \times 160}{\milli\meter} Eurocard, and the part costs
should be somewhere around \EUR{100}.

In \autoref{sec:hardware} the hardware design as well as the AD5933 are discussed.
\hyperref[sec:software]{Chapter \ref*{sec:software}} will describe the device firmware and the MATLAB interface
functions.
\hyperref[sec:results]{Chapter \ref*{sec:results}} will cover the results from measurements with the device.
A user's manual that shows the configuration and use of the device can be found in \autoref{sec:manual}.
Finally, in \autoref{sec:api} the extensible aspects of the firmware are described in detail, to give programmers an
idea of where they need to look when developing new features.

The PCB is designed using EAGLE\footnotemark{}. The layout is somewhat convoluted because it is manufactured in-house
and holes cannot be plated through, so each connection to a through-hole pin has to be made on the bottom side.
The device can easily be assembled by hand, a soldering microscope is useful though, particularly when soldering the
TQFP48 and LQFP48 packages.

\footnotetext{EAGLE PCB Design Software available from \url{http://www.cadsoft.de/}.}

The firmware for the device is written in C using the Eclipse C Development Tools\footnotemark{}, the MATLAB interface
functions are written in MATLAB.

\footnotetext{The code is standard C11 however, so any development environment can be used.}


\section{Getting the Files}

The complete source code, the EAGLE project files, as well as the \LaTeX{} documentation are available from the project
homepage at \url{http://feichti.net/impy/}. The project can be checked out of the Subversion repository, zipped source
code packages or compiled firmware images are available on request.
