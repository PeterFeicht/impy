\chapter{Programmer's Manual} \label{sec:api}

This part will be a short one, describing where programmers can start to extend the features of the device or
change the hardware configuration.
The main parts of the firmware have already been discussed in \autoref{sec:software}, so in this section
the extensible parts will be highlighted in a little more detail.


\section{Using the Virtual Console}

When writing an application that interfaces with the device (like the MATLAB scripts from \autoref{sec:matlab}),
the virtual console interface still needs to be used for device configuration.
To make non-interactive communication easier, there are two means by which the echoing of sent characters can be turned
off.

Each line that is sent can be preceded by an \texttt{@} character, thereby suppressing the echo of sent characters
for each line. This is the preferred method, since no permanent configuration is changed (which may lead to users
trying to use the device manually not seeing what they type).

The other method is to send the command \command{board set --echo=off} to turn echo off permanently. After sending
this command, no line prefix is needed (it is ignored if sent), however to turn echo on the command has to be sent
again with \command{--echo=on}.


\section{\texttt{main.h}}

This file declares the functions provided by the main module, some of the compile-time configuration options and the
HAL peripheral handles. This is most useful when writing additional interfaces, the console module for instance uses
the functions provided by the main module to configure the sweep parameters and board settings, get the board status
or start a sweep.

The main module provides functions directly reflecting the possible commands, like
\verb!Board_SetStartFreq(uint32_t freq)!, with a function for every command the virtual console interface accepts.
These functions also verify their arguments, so verification doesn't need to be implemented in every interface module.

The header file declares preprocessor constants for the GPIO ports used for the SPI peripheral.


\section{\texttt{mx\_init.c}}

This file contains the hardware initialization code. If additional peripherals need to be configured, this is the
place to add the initialization code. When additional modules are to be used, their respective HAL module needs
to be activated in \verb!stm32f4xx_hal_conf.h! as well and, when using Eclipse, the corresponding source file in
\verb!system/src/stm32f4-hal/! included in the build configuration.

The module-specific hardware configuration options (used GPIO pins, clock frequencies and the like) are mostly declared
in the header files of the modules using them.


\section{\texttt{console.h}}

This file declares the functions for the virtual console interface. It also declares a console interface structure
that is passed to the \verb!Console_ProcessLine! function, which holds pointers to functions used for communication.
This makes the console module reusable for different types of back end.

For example, both the USB VCP interface and a newly added Ethernet interface can use the virtual console module by
passing pointers to their respective communication functions. The console module then uses these pointers when it wants
to send a string or a data buffer, or when it wants to tell the back end that the current command has finished.

Currently, the callback functions for commands that don't finish immediately (calibration and temperature measurement)
are directly called from the main module. When interfaces other than the virtual console are to be added, these
calls need to be changed in order for the right interface to get a callback.


\section{Private API}

There are modules that are not supposed to be used directly by new types of interface or other extensions to the board.

The AD5933 driver module and EEPROM module header files declare the data structures used by these modules, but their
functions should not be called directly as these drivers are managed by the main module.

\subsection{EEPROM Data Structures}

However, the data structures declared in \verb!eeprom.h! may be extended when additional configuration data or settings
need to be stored on the EEPROM. The board configuration can be accessed directly by including \verb!main.h!.
For storing settings, the private functions \verb!UpdateSettings! and \verb!InitFromEEPROM! in \verb!main.c! should
be extended. A write of settings to the EEPROM can then be scheduled by calling the function \verb!MarkSettingsDirty!.

When adding additional data to the configuration and settings structures, care must be taken to adjust the size of the
\verb!reserved! fields, so the overall size of the structures doesn't change.

