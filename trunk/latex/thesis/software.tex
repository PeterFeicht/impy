\chapter{Software} \label{sec:software}

This part will describe the firmware of the device and the MATLAB functions.

As mentioned in \autoref{sec:hardware}, the spectrometer is based on the \enquote{STM32F4 Discovery} board, which has
an STM32F407 microcontroller and a programmer on board. On the PC side, besides an ordinary command line interface,
MATLAB can be used to control the device using some functions that are part of the provided software.


\section{Firmware}

The firmware is written in C and uses the \enquote{STM32CubeF4}
HAL\footnote{\textbf{H}ardware \textbf{A}bstraction \textbf{L}ayer, makes it easier to switch microcontrollers
without having to rewrite the device specific code.}
drivers to configure the microcontroller and control the peripherals.
It consists of four loosely coupled modules along with some support modules:
\begin{itemize}
	\item \verb!ad5933.c!: The AD5933 driver which takes care of communication with the AD5933 via \iic{} and collects
        the measured data.
  
  \item \verb!usbd_vcp_if.c!: The USB Virtual COM Port (VCP) interface definition, provides communication with the
        host computer.
  
  \item \verb!console.c!: Implements the virtual console that is used to control the device using a terminal
        program like \href{http://www.chiark.greenend.org.uk/~sgtatham/putty/}{PuTTY}.
  
  \item \verb!main.c!: The main module that holds the current configuration and provides an interface that is used
        by \verb!console.c! to control the basic functions of the device (e.g.\ starting a measurement, getting the
        current status, converting measured data).
\end{itemize}

Among the support modules are:
\begin{itemize}
	\item \verb!convert.c!: Converts measurement data to the different formats used for transfer via the USB connection.
  
  \item \verb!eeprom.c!: Used for communication with the on-board EEPROM, \verb!eeprom.h! declares the data
        structures used for storing the device configuration.
  
  \item \verb!util.c!: Provides functions for converting values (integers, IP addresses) to strings and vice versa.
\end{itemize}
