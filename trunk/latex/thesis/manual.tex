\chapter{User's Manual} \label{sec:manual}

This manual will describe the impedance spectrometer in detail and give step by step instructions for using it.


\section{Overview}

An overview of the device connections can be seen in Figure \ref{fig:device_connections}. Connector footprints for
unimplemented features are not labeled, the push buttons on the board are the same as those on the
STM32F4 Discovery board.

\begin{figure}[htb]
  \centering
    \includegraphics[width=\textwidth]{bilder/device_connections.jpg}
  \caption[Interface on the assembled impedance spectrometer]{
    Interface on the assembled impedance spectrometer:
    \begin{enumerate*}[label=\textbf{\arabic*}, itemjoin={{ -- }}]
      \item \label{itm:usb_prog}  USB programming and power connector
      \item \label{itm:usb_dev}   device USB connector
      \item \label{itm:leds}      status LEDs
      \item \label{itm:btn_rst}   reset button
      \item \label{itm:dc_jack}   DC power jack
      \item \label{itm:r_cal}     board with calibration resistors
      \item \label{itm:meas_out}  measurement output connections, sense (left) and force (right)
      \item \label{itm:meas_in}   measurement input connections (same as output)
      \item \label{itm:meas_gnd}  ground connections
      \item \label{itm:jump_idd}  $ I_\text{DD} $ measurement jumper
      \item \label{itm:jump_icc}  $ I_\text{CC} $ measurement jumper
      \item \label{itm:jump_vbus} $ V_\text{BUS} $ jumper
      \item \label{itm:r_fb}      feedback resistors
      \item \label{itm:r_att}     attenuation resistors
    \end{enumerate*} }
  \label{fig:device_connections}
\end{figure}

The USB programming connector \ref{itm:usb_prog} is used to program and debug the device, it can also be used to power
the device from a standard phone charger or similar supply with a Mini-B USB plug.
The device USB connector \ref{itm:usb_dev} is used to connect the device to a PC and control it.
The DC jack \ref{itm:dc_jack} is a standard \unit{2.5}{\milli\meter} jack (center positive) and can be used to
power the device with a \unit{5}{\volt} DC supply.

When using the device with an external power supply, or powered from the programming connector \ref{itm:usb_prog},
the $ V_\text{BUS} $ jumper \ref{itm:jump_vbus} may be disconnected to prevent the board from drawing power from the
device connector \ref{itm:usb_dev}.

The jumpers \ref{itm:jump_idd} and \ref{itm:jump_icc} can be used to measure the current consumption of the
microcontroller and all the other parts, respectively (note that there are two jumpers for $ I_\text{DD} $,
one on the STM32F4 Discovery board and one on the board, and care should be taken to use only one).

There are four status LEDs \ref{itm:leds}, three of which are used at this time:
\begin{itemize}
	\item \textbf{\color{blue} blue} will blink when the device is powered on, indicating that the firmware
    hasn't locked up,
  \item \textbf{\color{OliveGreen} green} will light up while a measurement is in progress, allowing the user
    to see when it has finished,
  \item \textbf{\color{red} red} will be turned on in case of an error in the firmware, this means the device
    should be reset using the reset button \ref{itm:btn_rst} and the steps leading up to the error should be
    documented, allowing the programmer to fix it.
\end{itemize}

\subsection{First Steps}

Before powering up the device for the first time, the following things should be checked:
\begin{itemize}
	\item the STM32F4 discovery board is properly connected,
  \item a board with calibration resistors is connected to the calibration pins \ref{itm:r_cal},
  \item the necessary feedback and attenuation resistors \ref{itm:r_fb} and \ref{itm:r_att} are soldered on,
  \item the current measurement jumpers \ref{itm:jump_idd} and \ref{itm:jump_icc} are connected
    (or a current meter of course),
  \item the $ V_\text{BUS} $ jumper \ref{itm:jump_vbus} is connected when no external power supply is used.
\end{itemize}

After connecting the device to a PC, any application that can access a serial port can be used to communicate with it.
Settings such as baud rate or parity don't matter.
When not using the MATLAB functions (see \autoref{sec:matlab}), the device is configured via a simple virtual console
interface using human readable commands. By typing \command{help}, a list of possible commands and their
explanations can be displayed.

The following steps assume the device is fitted with an EEPROM for configuration storage.
The first thing to do after assembling the device is to configure the fitted calibration and feedback resistors, as well
as possible attenuations and the coupling capacitor time constant using the \command{setup} command:
\begin{itemize}
	\item enter possible output voltage attenuations in the order they are selected with the attenuation multiplexer:
    \command{setup attenuation <values>...}
  \item enter feedback resistor values in the order they are connected to the feedback multiplexer in ohms:
    \command{setup feedback <values>...}
  \item enter calibration resistor values in the order they are soldered on the little resistor board (right to left):
    \command{setup calibration <values>...}
  \item enter the coupling capacitor time constant in ms (calculated by $ C_\text{coupl} \cdot \unit{1.1}{\kilo\ohm} $):
    \command{setup coupl <milliseconds>}
\end{itemize}
Typing \command{help setup} shows other options for the \command{setup} command, however none of them are currently
used since the respective peripherals are not yet implemented.

By typing \command{board info}, general information for the whole board is displayed, which includes the configured
resistors and attenuation values.


\section{Making Measurements}

After the first steps have been completed, measurements can be performed.

First, the unknown impedance(s) need to be connected to the output and input connectors \ref{itm:meas_out} and
\ref{itm:meas_in}. When a four-wire connection is not used, the two output and input pins, respectively, need to be
connected nonetheless for proper operation.

\subsection{Sweep Configuration}

Then, the sweep parameters need to be configured. This is done using the \command{board set} command with the following
parameters (only those that should be changed need to be sent):
\begin{itemize}
	\item \command{--start=FREQ} and \command{--stop=FREQ} set the start and stop frequency of the sweep in \hertz.
    Note that when setting start and stop frequency in a single command, an error can occur when the new start
    frequency is equal to or higher than the previous stop frequency. In this case you need to change the order in
    which start and stop options appear on the command line.
  \item \command{--steps=NUM} sets the number of frequency steps in a sweep, that is the number of times the frequency
    is incremented.
  \item \command{--settl=NUM} sets the number of settling cycles (see \autoref{sec:ad5933_proc}). The valid range is
    0--511 and can be extended 2 or 4 times by specifying a multiplier like \command{x2}. For example, valid values
    would be \command{20} or \command{40x2}, but not \command{512} (use \command{256x2} instead).
  \item \command{--voltage=RANGE} sets the output voltage range in \milli\volt. The possible ranges are determined by
    the configured attenuation values, they are one of the AD5933 output voltages (\unit{200}{\milli\volt},
    \unit{400}{\milli\volt}, \unit{1}{\volt} or \unit{2}{\volt}) divided by one of the attenuation values. For example,
    if attenuation values of 1 and 100 were configured, the possible voltage ranges would be 2000, 1000, 400, 200, 20,
    10, 4 and 2.
  \item \command{--gain=(on|off)} sets whether the input gain stage (PGA) is enabled. When enabled, the input signal is
    amplified by a factor of 5 prior to sampling.
  \item \command{--feedback=OHMS} sets the used feedback resistor, this needs to be one of the configured values.
\end{itemize}
The current value for any parameter can be displayed with the \command{board get} command, which accepts one of the
parameters and displays its value (for example \command{board get start} would display the start frequency in \hertz).
A quick overview of the current values can be displayed by typing \command{board get all}, which displays all sweep
parameters in a format suitable for parsing by PC software.

When selecting range settings, the following procedure should be used:
\begin{itemize}
	\item Set the output voltage low enough so the output current does not exceed \unit{10}{\milli\ampere} with the
    lowest impedance expected.
  \item Set the current feedback resistor to a value low enough so the input voltage stays below 3V for the highest
    output current expected.
  \item If necessary, the PGA can be enabled to further amplify the input voltage when high impedances are to be
    measured. For accurate results, the input signal should be close to 3V for the lowest impedance expected.
\end{itemize}

\subsection{Calibration}

Before measurements can be made the system has to be calibrated with a known impedance (see \autoref{sec:ad5933_proc}
on why this is necessary). After the range settings (start and stop frequency, output voltage, feedback resistor or
PGA gain) are changed, the \command{board calibrate} command is used to do that. The calibration impedance, specified
as the only parameter to the \command{board calibrate} command, should be in the same range as the impedance to be
measured for accurate results.

A recalibration is also necessary when the ambient temperature changes significantly.

Normally, calibration resistors are connected to the device with the small PCB \ref{itm:r_cal} mounted on the
calibration connectors. However, these connectors are just measurement connectors like \ref{itm:meas_out} and
\ref{itm:meas_in}, so calibration resistors can also be connected without a dedicated calibration resistor board.
\emph{Note that resistors should always be used for calibration, because the calibration impedance must not introduce
  a phase shift of its own.}
